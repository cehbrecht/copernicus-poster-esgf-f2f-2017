%----------------------------------------------------------------------------------------
%	PACKAGES AND OTHER DOCUMENT CONFIGURATIONS
%----------------------------------------------------------------------------------------

\documentclass[portrait,a0paper,fontscale=0.4]{baposter} % Adjust the font scale/size here

\usepackage{calc}
\usepackage{relsize}   % For \smaller
%\usepackage{multirow}
%\usepackage{rotating}
%\usepackage{bm}
\usepackage{url}       % For \url

\usepackage{graphicx}
%\usepackage{subcaption} % For floating figures

\usepackage{multicol} % Required for multiple columns
%\usepackage{wrapfig}
\setlength{\columnsep}{1.5em} % Slightly increase the space between columns
\setlength{\columnseprule}{0mm} % No horizontal rule between columns

\usepackage{tikz} % Required for flow chart
\usetikzlibrary{calc}

%\usepackage{times}
%\usepackage{helvet}
%\usepackage{bookman}
\usepackage{palatino}

%%% Global Settings %%%%%%%%%%%%%%%%%%%%%%%%%%%%%%%%%%%%%%%%%%%%%%%%%%%%%%%%%%%

\graphicspath{{images/}{draw.io/}}	% Root directory of the pictures
%\tracingstats=2			% Enabled LaTeX logging with conditionals

%%%%%%%%%%%%%%%%%%%%%%%%%%%%%%%%%%%%%%%%%%%%%%%%%%%%%%%%%%%%%%%%%%%%%%%%%%%%%%%%
% Save space in lists. Use this after the opening of the list
%%%%%%%%%%%%%%%%%%%%%%%%%%%%%%%%%%%%%%%%%%%%%%%%%%%%%%%%%%%%%%%%%%%%%%%%%%%%%%%%
\newcommand{\compresslist}{%
\setlength{\itemsep}{1pt}%
\setlength{\parskip}{0pt}%
\setlength{\parsep}{0pt}%
}

%%%%%%%%%%%%%%%%%%%%%%%%%%%%%%%%%%%%%%%%%%%%%%%%%%%%%%%%%%%%%%%%%%%%%%%%%%%%%%
%%% Begin of Document
%%%%%%%%%%%%%%%%%%%%%%%%%%%%%%%%%%%%%%%%%%%%%%%%%%%%%%%%%%%%%%%%%%%%%%%%%%%%%%

\begin{document}

%%%%%%%%%%%%%%%%%%%%%%%%%%%%%%%%%%%%%%%%%%%%%%%%%%%%%%%%%%%%%%%%%%%%%%%%%%%%%%
%%% Here starts the poster
%%%---------------------------------------------------------------------------
%%% Format it to your taste with the options
%%%%%%%%%%%%%%%%%%%%%%%%%%%%%%%%%%%%%%%%%%%%%%%%%%%%%%%%%%%%%%%%%%%%%%%%%%%%%%
% Define some colors

%\definecolor{lightblue}{cmyk}{0.83,0.24,0,0.12}
\definecolor{lightblue}{rgb}{0.145,0.6666,1}

%%
\begin{poster}%
  % Poster Options
  {
  % Show grid to help with alignment
  grid=false,
  % Number of columns
  columns=3,
  % Column spacing
  colspacing=1em,
  % Color style
  bgColorOne=white, % Background color for the gradient on the left side of the poster
  bgColorTwo=white, % Background color for the gradient on the right side of the poster
  borderColor=lightblue, % Border color
  headerColorOne=black, % Background color for the header in the content boxes (left side)
  headerColorTwo=lightblue, % Background color for the header in the content boxes (right side)
  headerFontColor=white, % Text color for the header text in the content boxes
  boxColorOne=white, % Background color of the content boxes
  boxColorTwo=lightblue,
  % Format of textbox
  textborder=roundedleft,
  % Format of text header
  eyecatcher=true, % Set to false for ignoring the left logo in the title and move the title left
  headerborder=closed, % Adds a border around the header of content boxes
  headerheight=0.10\textheight, % Height of the header
  headershape=roundedright, % Specify the rounded corner in the content box headers
  headershade=shadelr,
  headerfont=\Large\bf\textsc, %Sans Serif
  % textfont={\setlength{\parindent}{1.5em}},
  % boxshade=plain,
  % background=shade-tb,
  % background=plain,
  linewidth=2pt
  }
%%% Eye Cacther %%%%%%%%%%%%%%%%%%%%%%%%%%%%%%%%%%%%%%%%%%%%%%%%%%%%%%%%%%%%%%%
  {
      \includegraphics[width=10em]{copernicus-logo}
      \includegraphics[width=12em]{c3s-logo}
  }
%%% Title %%%%%%%%%%%%%%%%%%%%%%%%%%%%%%%%%%%%%%%%%%%%%%%%%%%%%%%%%%%%%%%%%%%%%
  {\sf\bf
          CP4CDS - A Compute Solution for\\ Copernicus Climate Change Service
  }
%%% Authors %%%%%%%%%%%%%%%%%%%%%%%%%%%%%%%%%%%%%%%%%%%%%%%%%%%%%%%%%%%%%%%%%%%
  {
    \vspace{1em} A. Stephens, S. Kindermann, S. Denvil, W. Som de Cerff\\
    {\smaller Centre for Environmental Data Analysis (CEDA),
    German Climate Compute Centre (DKRZ),\\
    Institut Pierre Simon Laplace (IPSL),
    Royal Netherlands Meteorological Institute (KNMI)\\[1em]
    \includegraphics[width=4em]{esgf-logo} ESGF F2F Meeting, San Francisco, 2017}
  }
%%% Logo %%%%%%%%%%%%%%%%%%%%%%%%%%%%%%%%%%%%%%%%%%%%%%%%%%%%%%%%%%%%%%%%%%%%%%
  {
    \begin{minipage}{10em}
      \includegraphics[width=10em]{ceda-logo} \\
      \includegraphics[width=10em]{ipsl-logo} \\
      \includegraphics[width=7em]{dkrz-logo}
    \end{minipage}
  }

%%%%%%%%%%%%%%%%%%%%%%%%%%%%%%%%%%%%%%%%%%%%%%%%%%%%%%%%%%%%%%%%%%%%%%%%%%%%%%
%%% Now define the boxes that make up the poster
%%%---------------------------------------------------------------------------
%%% Each box has a name and can be placed absolutely or relatively.
%%% The only inconvenience is that you can only specify a relative position
%%% towards an already declared box. So if you have a box attached to the
%%% bottom, one to the top and a third one which should be in between, you
%%% have to specify the top and bottom boxes before you specify the middle
%%% box.
%%%%%%%%%%%%%%%%%%%%%%%%%%%%%%%%%%%%%%%%%%%%%%%%%%%%%%%%%%%%%%%%%%%%%%%%%%%%%%

%%%%%%%%%%%%%%%%%%%%%%%%%%%%%%%%%%%%%%%%%%%%%%%%%%%%%%%%%%%%%%%%%%%%%%%%%%%%%%
\headerbox{Copernicus}{name=copernicus,column=0,row=0}{
%%%%%%%%%%%%%%%%%%%%%%%%%%%%%%%%%%%%%%%%%%%%%%%%%%%%%%%%%%%%%%%%%%%%%%%%%%%%%%
  \includegraphics[width=1.0\textwidth]{copernicus}

  \begin{itemize}\compresslist
    \item Copernicus is the European Union's earth observation programme
    \item Data from multiple sources: earth observation, satellites and in situ sensors
    \item Thematic areas: land, marine, atmosphere, \emph{climate change},
        emergency management, security
    \item Users: policymakers and public authorities
  \end{itemize}

}

%%%%%%%%%%%%%%%%%%%%%%%%%%%%%%%%%%%%%%%%%%%%%%%%%%%%%%%%%%%%%%%%%%%%%%%%%%%%%%
\headerbox{Climate Change Service (C3S)}{name=c3s,column=0,below=copernicus}{
%%%%%%%%%%%%%%%%%%%%%%%%%%%%%%%%%%%%%%%%%%%%%%%%%%%%%%%%%%%%%%%%%%%%%%%%%%%%%%
  \center
  \includegraphics[width=12em]{c3s-logo}
  \includegraphics[width=0.5\linewidth]{c3s-global-average-tas}
  \includegraphics[width=0.3\linewidth]{c3s-usage}

  \begin{itemize}\compresslist
    \item Information for monitoring and predicting climate change
    \item Helps to support adaptation and mitigation
    %\item The service will provide access to several climate indicators
    %  and climate indices for both the identified climate drivers and the expected climate impacts.
    \item Access to Global Climate Models projections using well-established
      metrics and manipulation tools and receive outputs
      tailored to their needs
    \item Products for coastal, water, insurance and energy sectors
  \end{itemize}

  Users benefits are:
  \begin{itemize}\compresslist
    \item no need to download and store large data sets
    \item data access from anywhere
    \item easily performing the same analysis for several datasets
    \item automatically generated metrics for indicating data sets quality
    \item logged commands to make work reproducible
    \item pre-defined functionalities for reducing programming work-load
    \item easier usability of climate model data by tailored tools
      for specific sectors (insurance, water, energy, coastal)
  \end{itemize}
}


%%%%%%%%%%%%%%%%%%%%%%%%%%%%%%%%%%%%%%%%%%%%%%%%%%%%%%%%%%%%%%%%%%%%%%%%%%%%%%
\headerbox{Climate Data Store (CDS)}{name=cds,column=0,span=1,below=c3s}{
%%%%%%%%%%%%%%%%%%%%%%%%%%%%%%%%%%%%%%%%%%%%%%%%%%%%%%%%%%%%%%%%%%%%%%%%%%%%%%
  \center
  \includegraphics[width=0.7\linewidth]{cds}

  % The Climate Data Store (CDS) will provide software (its toolbox) to allow users to develop their own applications,
  % making use of the CDS content to analyse, monitor and predict the patterns of both climate drivers and their impacts.
  % In this way, the CDS will be at the heart of the Copernicus Climate Change Service (C3S)
  % and will provide access to climate observations, global and regional climate reanalyses,
  % global and regional climate projections and seasonal via an easy-to-use web interface.

  \begin{itemize}\compresslist
    \item A climate data store will contain the geophysical information needed to analyse the climate change indicators
      in a consistent and harmonised way.
    \item This will combine the functions of a distributed data centre with a set of services and facilities
      for users and content developers.
    \item The store will provide data resources and computing facilities that can be utilised, for example,
      to develop improved climate reanalyses and seasonal forecasts.
  \end{itemize}
}


%%%%%%%%%%%%%%%%%%%%%%%%%%%%%%%%%%%%%%%%%%%%%%%%%%%%%%%%%%%%%%%%%%%%%%%%%%%%%%
\headerbox{References}{name=references,column=0,below=cds}{
%%%%%%%%%%%%%%%%%%%%%%%%%%%%%%%%%%%%%%%%%%%%%%%%%%%%%%%%%%%%%%%%%%%%%%%%%%%%%%
  \smaller													% Make the whole text smaller
  \vspace{-0.4em} 									% Save some space at the beginning
  \bibliographystyle{plain}					% Use plain style
  \renewcommand{\section}[2]{\vskip 0.05em}		% Omit "References" title
  \begin{thebibliography}{1}							% Simple bibliography with widest label of 1
    \itemsep=-0.01em										% Save space between the separation
    \setlength{\baselineskip}{0.4em}					% Save space with longer lines
    \bibitem{prevWork1} Laszlo Gulyas, Richard Legendi: \emph{Effects of Sample Duration on Network Statistics in Elementary Models of Dynamic Networks}, International Conference on Computational Science, Singapore (2011)
    \bibitem{prevWork2} Laszlo Gulyas, Susan Khor, Richard Legendi and George Kampis \emph{Cumulative Properties of Elementary Dynamic Networks}, The International Sunbelt Social Network Conference XXXI (2011)
  \end{thebibliography}
}

%%%%%%%%%%%%%%%%%%%%%%%%%%%%%%%%%%%%%%%%%%%%%%%%%%%%%%%%%%%%%%%%%%%%%%%%%%%%%%
\headerbox{Outlook}{name=outlook,column=0,span=1,below=references}{
%%%%%%%%%%%%%%%%%%%%%%%%%%%%%%%%%%%%%%%%%%%%%%%%%%%%%%%%%%%%%%%%%%%%%%%%%%%%%%
  \begin{itemize}\compresslist
    \item C3S overview and C3S Climate Data Store (CDS)
    \item The CP4CDS project overview: ESGF Data Nodes, Index Nodes and CP4CDS Compute Nodes,
      load-balancing across sites, QC'd CMIP5 data.
    \item Requirement to deploy code from other projects/groups into the CP4CDS Compute Node, such as ESMValTool.
    \item Chosen technology: Birdhouse WPS environment
    \item Why Birdhouse? python, open source, actively developed, good design, client/security components, PyWPS extensions.
    \item SDDS/Compute Node overview: including PyWPS cluster configuration for scalability.
    \item SDDS workflow: template, GitHub fork etc
    \item Status and next steps
  \end{itemize}
}

%%%%%%%%%%%%%%%%%%%%%%%%%%%%%%%%%%%%%%%%%%%%%%%%%%%%%%%%%%%%%%%%%%%%%%%%%%%%%%
\headerbox{Climate Projections for the Copernicus Data Store (CP4CDS)}{name=cp4cds,column=1,span=2,row=0}{
%%%%%%%%%%%%%%%%%%%%%%%%%%%%%%%%%%%%%%%%%%%%%%%%%%%%%%%%%%%%%%%%%%%%%%%%%%%%%%
   % The Copernicus Climate Data Store (CDS) will interact with remote data and compute services,
   % amongst which will be those providing global climate projections.

   % An Earth System Grid Federation (ESGF) node and concomitant compute services
   % providing access to a coherent collection of quality assured climate projections.

   % CP4CDS will be delivered by a consortium of three of the leading European institutes

   % deliver the required data and compute services in a flexible and scalable environment with,
   % by the end of the project, high standards of reliability and performance

   % The CP4CDS system will be based on a geographically distributed highly available set of data and compute services.
   % Data will be available at CEDA, IPSL, and DKRZ, with replicas synchronised using the Synda software package.

   % A Software Dependency Deployment Solution (SDDS) consisting of code and library repositories
   % with continuous integration and a release cycle
   % to support the development of codes which exploit the data and compute interfaces offered by CP4CDS;

   % A hosted compute service with direct access on the local file system to the entire ESGF cache

   % A software toolbox to be used by those developing codes for interfacing to CP4CDS,
   % including the esgf_pyclient python library for searching ESGF for data and the
   % Synda python library for managing data movement and processing.

   % An extended ESGF sub-system populated and configured to support the Copernicus Data Store
   % identifying, acquiring and quality controlling the relevant data
   % data nodes - consisting of vanilla ESGF index and data nodes
   % compute nodes - based on the Web Processing Service (WPS)
   % SDDS - Software Dependency Deployment Solution to support the development of codes
   % All web-service access will require users to be authenticated and authorised
   % Federated data nodes
   % Data store with CMIP5 and CORDEX data - selected for C3S and quality checked

   \begin{minipage}{0.3\textwidth}
     \includegraphics[width=0.8\linewidth]{cp4cds}
   \end{minipage}
   \begin{minipage}{0.7\textwidth}
     \begin{itemize}\compresslist
       \item Providing climate projections to the Climate Data Store of the
        Climate Change Service portal hosted at ECMWF, UK
       \item Data Node - consisting of vanilla \emph{Earth System Grid Federation} (ESGF) index and data node
       \item Compute Node - providing compute facilities using the Web Processing Service (WPS) standard interface
       \item Processing Backend - \emph{external software toolbox} to analyse climate model projections
       \item Climate Projections in filesystem cache - selected for C3S and \emph{quality checked}
       \item Replication - \emph{Synda} Python library for managing data movement
     \end{itemize}
     \vspace{2em}
  \end{minipage}
  \begin{minipage}{0.3\textwidth}
    \includegraphics[width=0.8\linewidth]{cp4cds-federated}
  \end{minipage}
  \begin{minipage}{0.7\textwidth}
    \begin{itemize}\compresslist
      \item Geographically distributed highly available set of data and compute services
      \item Federated between the leading European institutes: CEDA, IPSL and DKRZ
      \item Using load-balancing across sites / failover strategy
      \item CEDA is the default node, IPSL or DKRZ take over service when CEDA is not available
    \end{itemize}
    \vspace{2em}
  \end{minipage}
  \begin{minipage}{0.3\textwidth}
    \includegraphics[width=0.8\linewidth]{cp4cds-toolbox}
  \end{minipage}
  \begin{minipage}{0.7\textwidth}
    \begin{itemize}\compresslist
      \item Requirement to deploy code from other projects/groups into the CP4CDS Compute Node, such as ESMValTool.
      \item Toolbox with a \emph{Software Dependency and Deployment Solution} (SDDS)
      \item Supporting the development of codes which exploit the data and compute interfaces offered by CP4CDS
      \item Used to set-up CP4CDS nodes for CMIP5 (global) and CORDEX (regional) climate projections with
        a specific analysis toolbox.
    \end{itemize}
    \vspace{2em}
  \end{minipage}
 }

%%%%%%%%%%%%%%%%%%%%%%%%%%%%%%%%%%%%%%%%%%%%%%%%%%%%%%%%%%%%%%%%%%%%%%%%%%%%%%
\headerbox{Architecture}{name=arch,column=1,span=2,below=cp4cds}{
%%%%%%%%%%%%%%%%%%%%%%%%%%%%%%%%%%%%%%%%%%%%%%%%%%%%%%%%%%%%%%%%%%%%%%%%%%%%%%
  \begin{minipage}{0.3\textwidth}
    \includegraphics[width=\linewidth]{cp4cds-wps}
  \end{minipage}
  \begin{minipage}{0.7\textwidth}
    \begin{itemize}\compresslist
      \item A WPS request (HTTP GET/POST) comes from a WPS client.
      \item The Nginx/Gunicorn combination delegates the request to the PyWPS WSGI application
      \item Gunicorn - spawns several workers to use the available CPUs on a single compute node
      \item PyWPS - Python implementation of OGC Web Processing Standard
      \item Supervisor - used to start/stop and monitor services
      \item Processing outputs and status documents are web accessible by the Nginx file-service
      \item Token based access control to WPS service
      \item WPS Processes are defined for ESMValTool diagnostics
      \item ESMValTool has read- only access to the climate data pool on
        file-system with CMIP5 climate model and observational data.
      \item Using PyWPS scheduler extension (Slurm, GridEngine) to run process on a compute-cluster for scalability
    \end{itemize}
  \end{minipage}
  \vspace{1em}

  % \begin{minipage}{0.3\textwidth}
  %   \includegraphics[width=\linewidth]{pywps-scheduler-extension}
  % \end{minipage}
  % \begin{minipage}{0.7\textwidth}
  %   \begin{itemize}\compresslist
  %     \item PyWPS Scheduler Extension
  %   \end{itemize}
  % \end{minipage}

  \begin{minipage}{0.3\textwidth}
    \includegraphics[width=0.2\linewidth]{conda}
    \includegraphics[width=0.1\linewidth]{ansible}
    \includegraphics[width=0.4\linewidth]{chaos}
  \end{minipage}
  \begin{minipage}{0.7\textwidth}
    \begin{itemize}\compresslist
      \item Using the Conda package manager to setup an environment with all used software components.
      \item Using Buildout/Ansible to setup a WPS (PyWPS) with all services (Supervisor, Gunicorn, Nginx)
        and configuration files.
      \item ``Managing the Chaos''
    \end{itemize}
  \end{minipage}
  \vspace{1em}

  \begin{minipage}{0.3\textwidth}
    \includegraphics[width=0.2\linewidth]{docker}
    \includegraphics[width=0.4\linewidth]{cp4cds-docker-cloud}
  \end{minipage}
  \begin{minipage}{0.7\textwidth}
    \begin{itemize}\compresslist
      \item A Dockerfile is generated using the Buildout setup for each WPS service
      \item Docker images are build automatically on Docker Cloud
    \end{itemize}
  \end{minipage}
  \vspace{1em}

  % \begin{minipage}{0.3\textwidth}
  %   \includegraphics[width=\linewidth]{esmvaltool}
  % \end{minipage}
  % \begin{minipage}{0.7\textwidth}
  %   \begin{itemize}\compresslist
  %     \item A community diagnostic and performance metrics tool for routine evaluation of Earth system models in CMIP
  %     \item Collection of diagnostics implemented in Python and NCL
  %   \end{itemize}
  % \end{minipage}
}


%%%%%%%%%%%%%%%%%%%%%%%%%%%%%%%%%%%%%%%%%%%%%%%%%%%%%%%%%%%%%%%%%%%%%%%%%%%%%%
\headerbox{SDDS - Software Deployment Dependency Solution}{name=sdds,column=1,span=2,below=arch}{
%%%%%%%%%%%%%%%%%%%%%%%%%%%%%%%%%%%%%%%%%%%%%%%%%%%%%%%%%%%%%%%%%%%%%%%%%%%%%%
  \begin{minipage}{0.25\textwidth}
    \includegraphics[width=\linewidth]{sdds}
  \end{minipage}
  \begin{minipage}{0.75\textwidth}
    \begin{itemize}\compresslist
      \item Compute Node is capable of hosting processing codes from external projects
      \item SDDS - consists of a software environment and application, managed through a GitHub repository,
        which includes a basic template of a working WPS service.
      \item The template uses a Conda ``environment'' to record the software dependencies
      \item Conda is the tool to build reproducible software environments
      \item Docker is used to provide the Compute Node through containers
    \end{itemize}
  \end{minipage}
  \vspace{2em}

  \begin{minipage}[t]{0.45\textwidth}
    \includegraphics[width=\linewidth]{sdds-workflow} \\
    Workflow for creating a new WPS compute service based on the SDDS template.
  \end{minipage}
  \hspace{5em}
  \begin{minipage}[t]{0.45\textwidth}
    \includegraphics[width=\linewidth]{sdds-update-cycle} \\
    Update cycle for a new WPS compute service. \\
  \end{minipage}

  % \begin{itemize}\compresslist
  %   \item Compute Node capable of hosting processing codes from external projects
  %   \item Compute Node will employ the Web Processing Service (WPS) standard
  %   \item CP4CDS project is developing a system for managing different software environments in a re-usable manner
  %   \item consists of code and library repositories to support the management,
  %     testing and deployment of processing code that will be made available via the production service
  %   \item It consists of a software environment and application, managed through a GitHub repository,
  %     which includes a basic template of a working WPS process.
  %     The template uses a Conda "environment" to record the software dependencies
  %   \item Conda as the tool to build reproducible software environments
  %   \item intend to employ Docker to provide the Compute Node through containers
  %   \item "production" environment run by the CP4CDS partners: STFC CEDA, IPSL and DKRZ.
  % \end{itemize}
}


\end{poster}

\end{document}
